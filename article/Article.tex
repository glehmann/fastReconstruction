%
% Complete documentation on the extended LaTeX markup used for Insight
% documentation is available in ``Documenting Insight'', which is part
% of the standard documentation for Insight.  It may be found online
% at:
%
%     http://www.itk.org/

\documentclass{InsightArticle}


%%%%%%%%%%%%%%%%%%%%%%%%%%%%%%%%%%%%%%%%%%%%%%%%%%%%%%%%%%%%%%%%%%
%
%  hyperref should be the last package to be loaded.
%
%%%%%%%%%%%%%%%%%%%%%%%%%%%%%%%%%%%%%%%%%%%%%%%%%%%%%%%%%%%%%%%%%%
\usepackage[dvips,
bookmarks,
bookmarksopen,
backref,
colorlinks,linkcolor={blue},citecolor={blue},urlcolor={blue},
]{hyperref}


%  This is a template for Papers to the Insight Journal. 
%  It is comparable to a technical report format.

% The title should be descriptive enough for people to be able to find
% the relevant document. 
\title{Improving performance of morphological reconstruction}

% Increment the release number whenever significant changes are made.
% The author and/or editor can define 'significant' however they like.
\release{0.00}

% At minimum, give your name and an email address.  You can include a
% snail-mail address if you like.
\author{Richard Beare}
\authoraddress{Department of Medicine, Monash University, Australia}

\begin{document}
\maketitle

\ifhtml
\chapter*{Front Matter\label{front}}
\fi


\begin{abstract}
\noindent
Morphological reconstruction may be implemented in a number of
different ways. ITK has an iterative method and a non iterative
method. This article compares the performance of another non iterative
method and finds a significant improvement.
\end{abstract}

\tableofcontents

\section{Introduction}
Morphological reconstruction by erosion or dilation, also referred to
as geodesic erosion or dilation, operate on a marker image and a mask
image. In the case of dilation the marker image is dilated by an
elementary structuring element and the pixelwise minimum of the result
and the mask is computed. This procedure can be run for a fixed number
of iterations or until convergence.

The original implementation of morhpological reconstruction in ITK
found in {\bf itkGrayscaleGeodesicDilateImageFilter} and {\bf
itkGrayscaleGeodesicErodeImageFilter} provides a direct implementation
of the approach outlined above. Unfortunately this approach is very
time consuming when the aim is to run until convergence.

A faster, non iterative implementation, has been provided in {\bf
itkReconstructionByDilationImageFilter} and {\bf
itkReconstructionByErosionImageFilter}. These filters implement the
algorithm described in \cite{Robinson2004a}.

This report investigates the performance of another algorithm
described in \cite{Vincent93a} and demonstrates another significant
improvement in performance.

\section{Hybrid algorithm}
The hybrid algorithm for morphological reconstruction described in \cite{Vincent93a} is:

\begin{itemize}
\item I : mask image
\item J : marker image, defined on domain $D_I$
\item scane $D_I$ in raster order:
   \begin{itemize}
	\item Let $p$ be the current pixel;
	\item $J(p) = (max\{J(q), q \in N^+_G(p) \cup \{p\}\}) \wedge I(p)$
   \end{itemize}
\item scane $D_I$ in anti-raster order:
   \begin{itemize}
	\item Let $p$ be the current pixel;
	\item $J(p) = (max\{J(q), q \in N^-_G(p) \cup \{p\}\}) \wedge I(p)$
	\item If there exist $q \in  N^-_G(p)$ such that $J(q) < J(p)$ and $J(q) < I(q)$ then {\em fifo\_add(q)}
   \end{itemize}
\item Propagation step: While (!fifo\_empty)
   \begin{itemize}
	\item p = fifo\_first()
        \item For every pixel $q \in N_G(p)$:
        \begin{itemize}
	\item If $J(q) < J(p)$ and $I(q) \ne J(q)$ then

	      $J(q) = \min\{J(p),I(q)\}$

         	{\em fifo\_add(q)}
		
	\end{itemize}
   \end{itemize}
\end{itemize}

where $N_G(p)$ denotes the neighborhood of $p$ on a grid $G$, with
$N^+$ and $N^-$ denoting the set of neighbors reached before and after
$p$ during a raster scan. $\wedge$ is the pointwise minimum operator.
\section{Implementation details}
The implementation of this algorithm is provided by {\bf
itkMorphologicalReconstructionImageFilter}, which takes functor
template arguments to provide either erosion or dilation
reconstruction filters. The reconstruction by erosion filter is {\bf
MorphologicalReconstructionErosionImageFilter} and the reconstruction
by dilation filter is {\bf
MorphologicalReconstructionDilationImageFilter}. These filters are
verified by comparison with the existing implementation using a series
of tests in which regional extrema are supressed. The tests use {\bf
itkHMinimaImageFilter2} and {\bf itkHMaximaImageFilter2} which are
versions of {\bf itkHMinimaImageFilter} and {\bf
itkHMaximaImageFilter} that have been modified to use the new
reconstruction filter.

A typical requirement is that the mask must always be greater than or
equal to the marker image. In the new filter this restriction is
enforced implicitly.

\subsection{Optimization}
Profiling of the initial implementation suggested that the boundary
checks were contributing a significant overhead, as would be
expected. An attempt was made to use the face calculator to improve
performance without increasing the size of the image (these
modifications are present in the code and may be activated by defining
the symbole ``FACE'' in the header file). Unfortunately the algorithm
is strongly dependent on visit order and operating on face and body
regions changes the visit ordering and produces incorrect
results. This strategy may be feasible with much more careful analysis
of the region borders.

A much simpler optimization is to pad the image. This option can be
activated by defining the ``COPY'' macro in the header file and
provides a considerable improvement in execution speed at the cost of
greater memory use. Someone should probably check my use of the pad
and crop filters to make sure that the mini-pipeline conforms to
standards.
\section{Performance tests}
The first column indicates the connectivity - 0 is face connected, 1
is fully connected.

2D image without optimization:
\begin{verbatim}
> ./perf2D images/cthead1.png
#F      Robinson        Vincent
0       0.1303          0.0222
1       0.13585         0.0273
\end{verbatim}

3D image without optimization:
\begin{verbatim}
> ./perf3D images/ESCells.hdr
#F      Robinson        Vincent
0       19.947  	3.795
1       23.771  	6.6095
\end{verbatim}

2D images with edge padding:
\begin{verbatim}
>  ./perf2D images/cthead1.png
#F      Robinson        Vincent
0       0.1394          0.0192
1       0.1444          0.02185
\end{verbatim}
3D image with edge padding:
\begin{verbatim}
> ./perf3D images/ESCells.hdr
#F      Robinson        Vincent
0       20.468          2.807
1       24.218          4.016
\end{verbatim}

\section{Conclusion}
The new implemenation of the Vincent algorithm is between 5 and 8
times faster than the Robinson algorithm, depending on the choice of
connectivity and image padding optimization. Optimization by padding
the boundary produced a significant saving for the 3D dataset used in
testing. The saving was not as significant for 2D data.



\bibliographystyle{plain}
%\bibliography{InsightJournal}
\bibliography{Article}
%\nocite{ITKSoftwareGuide}

\end{document}

